\section{Methodology}
%What was done.
As indicated by the posed research queries in Section \ref{sec_res_q},
this project consisted of two distinct development tasks:

\begin{itemize}
\item Implementation of a SLAM solution on a Thymio II robot.
\item Parallelisation of the existing Thymio simulator.
\end{itemize}

Due to the loose coupling of the simulator and the SLAM implementation, each of
the development tasks could be conducted in parallel by individual project
members.
The methodology and development techniques of each deliverable is detailed
below.

\subsection{SLAM Solution}
%What was implemented on the robot and limitations of the platform.
As has been previously detailed, SLAM solutions consists of three main
operations:
\begin{itemize}
    \item Robotic movement
    \item Sensor observation
    \item Map update
\end{itemize}
The three SLAM steps provided a simple and intuitive manner to divide the
implementation of an algorithmic solution of the SLAM problem.

\subsubsection{Robotic Movement}
The navigational portion of a SLAM solution requires a robot to not only move
through an area, but to also keep an approximate record of how each movement
has altered it's position in the environment.

There are two main approaches to allow a robot to estimate is position
without external aid; odometry and dead-reckoning.

Odometry is considered the preferable option for estimated localisation, as
the motion sensors required for implementing the feature can be extremely
reliable, resulting is very accurate estimations.
The alternative solution to onboard localisation estimation is via
dead-reckoning - where location is estimated based upon the speed that the
robot was travelling.
The basic dead-reckoning calculation is identical to the kinematic
distance formula:

\[ D = V \times T\]

Where the distance travelled \(D\) is equal to the product of velocity of
movement \(V\) and the time spent travelling \(T\).

As the Thymio II's wheels were not equipped with any form of rotary encoders,
native odometery was not possible, and thus dead-reckoning had to be used for
localisation estimation.
Though Thymio systems have been utilised as testing platforms in multiple
studies, including SLAM solutions that utilise dead-reckoning, the Thymio II
system does not include dead-reckoning `out-of-the-box' and thus a bespoke
implementation had to be developed.
Investigation into previous implementations of dead-reckoning on a Thymio
system reviled little usable, or adaptable software that could be utilised for
this project.
The only code relating to Thymio II estimated location was found in a
navigation driver plug-in for the Robot Operating System (ROS) - an open-source
middle-ware for robot control.
As the Thymio II used in this study was not utilising ROS as system driver,
the navigation plug-in could not be implemented directly - but still provided
significant insight into how the bespoke dead-reckoning could be implemented.

The ROS Thymio II navigation driver indicated that a unmodified Thymio II
robot had a speed coefficient - or hypothetical speed of 2.93.
Ad hoc experimentation using the Aseba simulator revealed that ROS navigation
module's estimated speed coefficient was accurate, but this value was not
representative of the real-world robot.
The Thymio II robot utilises an integrated battery to power its self, and
during operation the internal voltage of the battery is drained.
Unfortunately, due to the lack of power regulation, as the voltage of the
internal battery dropped, so to did the speed of the robots wheels.
The inconsistent speed wheel at which the wheels operated meant that the
constant speed coefficient extracted from the ROS package could not be used to
accurately determine the location of the robot - as the longer that the robot
was mobile, the slower its wheels would operate, resulting in more inaccurate
predictions as time moved on.




\subsection{Simulator}
How the simulator was implemented/improved.
