\section{Introduction} \label{sec_introduction}
This report serves to document and present the processes, methods, and
technologies utilised in the development of a combined real-world and
simulated implementation of a robotic Simultaneous Location and Mapping
system - allowing a physical robot to map environment, .


\subsection{Main Topics}
\subsubsection{SLAM}
Simultaneous Location and Mapping (SLAM) is computational problem relating
to the tracking of an agent's location in an environment while concurrently
constructing a map of the environment that contains the agent.

The primary difficulty to producing a solution to the SLAM problem is
that SLAM is a circular problem: in order to accurately determine a location
within an area, a map of said area is required - and in order to create a map
of environment, an agent already needs to be aware of it's location in the
environment.

A number of algorithms and approaches have been theorised and developed to
solve the problem of SLAM.
The choice of which SLAM solution implementation to utilise depends heavily
on the environment being mapped, the equipment available to the agent, and the
manner in which the map is to be produced.
Further discussion of SLAM and SLAM solutions is provided in Section REF HERE!


\subsubsection{Routing}
Routing problems consists of a simple premise: ``What is the optimal path to
take to travel from point A to B''.
The main difficulties present in the routing problem domain typically consist of
defining what 'optimal' means, and selecting the single best solution from the
huge array of possible answers.

In terms of optimisation, the best solution may be the path that takes the
least time, covers the smallest amount of distance, requires the less
resources, or any combination of the previous metrics.

Much like SLAM, a wide range of algorithms and formula have been researched
and evaluated to solve the routing problem - a selection of which are
discussed in Section REF HERE!


\subsubsection{Robotic Simulation}
Robotic simulators allow a robot's hardware or behaviour to be tested and
evaluated without relying upon the physical robot.

Regardless of the type of simulator utilised, the technology allows for
consistent tests to be conducted, and the testing parameters to be altered at
a reduced cost - both in terms of time and hardware requirements.


\subsubsection{Research questions}
The main questions the project addressed.


\subsection{Report Structure}
This report is divided appropriately;

A background review and analysis of existing literature, white papers, and
documentation; in order to both prove the feasibility of, and justify this
project and the techniques and technologies utilised in its execution.

A description of the methods and technologies utilised during the development
and testing of the system.

Details of the metrics used to evaluate the project.

Presentation of findings made during the implantation or testing.

An evaluation of the techniques used, and the results of the project.

A summery concluding the key findings of the project and potential future
work.

The personal reflections of the authors, their opinions of project, and any
improvements or changes that they have observed in hindsight.
