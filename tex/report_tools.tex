\section{Tools and Technologies}
%List and discuss the hard/software utilised in the project.
As is the case with many hardware centric studies, the choice of technologies
utilised by a project can drastically alter it outcome.
This project utilised a range of technologies and tools to implement the
proposed system, the details of which are provided here.

\subsection{Robot Platform}
A key concept of this study was the implementation of a SLAM solution
utilising inexpensive or primitive hardware components.
A Thymio II wireless robot was utilised for the implementation and testing of
the project's SLAM solution.
The Thymio II was chosen for this project due to its cost (typically available
for under £150) and its availability  - Edinburgh Napier University had
previously used the Thymio robots in a previous project.

The key features of the Tymio II wireless system are as follows:
\begin{itemize}
\item Nine infrared proximity sensors (five front mounted, two rear mounted,
and two ground facing), with an approximate range of 10cm.
\item 2.4GHz wireless module.
\item 3.7V, 1500mAh, Li-Po rechargeable battery.
\end{itemize}
Full specification of the Thymio II is visible at Tymio's website
\cite{thymio}.

While the robot's cost point and accessibility were ideal for this
experimentation, there did exist a number of concerns in regards to Thymio
II's hardware capabilities.

A major concern was the robot's lack of proprioceptive or odometric sensors -
sensors that a robotic agent can utilise to detect how far, or at what angle
it has moved.
A common odometry implementation is the use of rotary encoders on a robotic
agent's wheel.
The encoders can calculate the stance that the robot has travelled based on
the circumference of the wheels, and the number of rotations the wheels have
made.
While odometers is not required for the implementation of a SLAM solution, the
approximated localisation can significantly reduce the complexity of the
overall system.


The second major concern in regards to the Thymio II' specification was the
operational range of its infrared sensors.
Exteroceptive sensors, or external state sensors detect the environment around
them.
There are a range of such sensors that can be utilised in SLAM solutions,
including but not limited to; laser scanners, ultrasonic sensors, infrared
detectors, and cameras.
A mobile exteroceptive sensor is the minimum requirement to implement a SLAM
solution, however the operational accuracy and range of said sensor can
greatly alter the challenge of implementation.
The Thymio II's detection of range of 10cm was concerning, but a number of
localisation and navigation studies had been conducted previously, utilising
the Thymio II, lessening the initial alarm in regards to the sensors fidelity
\cite{Wang2016}.
