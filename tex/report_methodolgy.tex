\section{Methodology}
%What was done.
As indicated by the posed research queries in Section \ref{sec_res_q},
this project consisted of two distinct development tasks:

\begin{itemize}
\item Implementation of a SLAM solution on a Thymio II robot.
\item Parallelisation of the existing Thymio simulator.
\end{itemize}

Due to the loose coupling of the simulator and the SLAM implementation, each of
the development tasks could be conducted in parallel by individual project
members.
The methodology and development techniques of each deliverable is detailed
below.

\subsection{SLAM Solution}
%What was implemented on the robot and limitations of the platform.
As has been previously detailed, SLAM solutions consists of three main
operations:
\begin{itemize}
    \item Robotic movement
    \item Obstacle observation
    \item Map update
\end{itemize}
The three SLAM steps provided a simple and intuitive manner to divide the
implementation of an algorithmic solution of the SLAM problem.

\subsubsection{Robotic Movement}
The navigational portion of a SLAM solution requires a robot to not only move
through an area, but to also keep an approximate record of how each movement
has altered it's position in the environment.

There are two main approaches to allow a robot to estimate is position
without external aid; odometry and dead-reckoning.

Odometry is considered the preferable option for estimated localisation, as
the motion sensors required for implementing the feature can be extremely
reliable, resulting is very accurate estimations.
The alternative solution to onboard localisation estimation is via
dead-reckoning - where location is estimated based upon the speed that the
robot was travelling.
The basic dead-reckoning calculation is identical to the kinematic
distance formula:

\[ D = V \times T \]

Where the distance travelled \(D\) is equal to the product of velocity of
movement \(V\) and the time spent travelling \(T\).

As the Thymio II's wheels were not equipped with any form of rotary encoders,
native odometery was not possible, and thus dead-reckoning had to be used for
localisation estimation.
Though Thymio systems have been utilised as testing platforms in multiple
studies, including SLAM solutions that utilise dead-reckoning, the Thymio II
system does not include dead-reckoning `out-of-the-box' and thus a bespoke
implementation had to be developed.
Investigation into previous implementations of dead-reckoning on a Thymio
system reviled little usable, or adaptable software that could be utilised for
this project.
The only code relating to Thymio II estimated location was found in a
navigation driver plug-in for the Robot Operating System (ROS) - an open-source
middle-ware for robot control.
As the Thymio II used in this study was not utilising ROS as system driver,
the navigation plug-in could not be implemented directly - but still provided
significant insight into how the bespoke dead-reckoning could be implemented.

The ROS Thymio II navigation driver indicated that a unmodified Thymio II
robot had a speed coefficient - or hypothetical speed of 2.93.
Ad hoc experimentation using the Aseba simulator revealed that ROS navigation
module's estimated speed coefficient was accurate, but this value was not
representative of the real-world robot.
The Thymio II robot utilises an integrated battery to power its self, and
during operation the internal voltage of the battery is drained.
Unfortunately, due to the lack of power regulation, as the voltage of the
internal battery dropped, so to did the speed of the robots wheels.
The inconsistent speed wheel at which the wheels operated meant that the
constant speed coefficient extracted from the ROS package could not be used to
accurately determine the location of the robot - as the longer that the robot
was mobile, the slower its wheels would operate, resulting in more inaccurate
predictions as time moved on.

A second issue regarding robot's mobility was also discovered during the
experimentation process - the Thymio II's inability to travel in a straight
line.  
The Thymio robotic platform's movement operations are controlled by issuing
commands to change the speed and direction of each wheel individually - by
setting both wheels to the same speed and direction the robot should move in a
straight line.
Experimentation revelead that left wheel of the Thymio II utilised in this
project was unable to match the same speed as the right wheel - causing the
robot to curve leftwards when attempting to move directly forwards.
The cause for this issue was unapparent, but a number of possible reasons were
considered:

\begin{itemize}
\item Friction between the wheel and robot chassis.
\item Unbalanced power regulation.
\item Poor production tolerances.
\end{itemize}

A theorised solution was to limit the speed of the rightmost wheel to
percentage of the left.
While theoretically sound, the proposed solution proved futile, as the speed of
the robot's wheels were coupled non-linearly to the voltage of the internal
battery - resulting in the robot once again curving as the maximum speed of
the robot slowed.

While it was not possible to implement a reliable dead-reckoning
system on the physical robot, there was still potential for the estimated
localisation to function correctly within the perfect conditions of the
simulated environment, and thus development moved to the next stage of the
SLAM problem.

\subsubsection{Obstacle Observation}
To be a valid SLAM solution, a robotic agent is required to observe the
environment around its self in order to identify landmarks or features which
can be utilised by the robot to localise its self in the global environment.

As outlined in the \editnote{REF-TOOLS-SECTION!}, the Thymio II robot featured five
forward facing infrared sensors, with a documented effective range of \~{}10cm.
Preliminary experimentation was conducted to verify that the Thymio's
documentation was accurate.
Investigation revealed that the operational tolerances of the Thymio's sensors
was not consistent, with the majority of the infrared sensors only able to
consistently detect objects within a range of \~{}6-8cm.
The extremely limited range of the sensors made the task of object detection
excessively complex, as a robot would not be able to observe environmental
features without moving within close proximity of a map obstacle.
The short range of the sensors also limited the execution of the third SLAM
solution step; the updating of the agent's map.

\subsubsection{Map Update}
The final step of this projects proposed SLAM solution was to update the
robotic agent's approximated map of the environment.
The process of updating a map consisted of two steps; determining if the
obstacle currently being observed was already recorded in the map, and in cases
where the obstacle has not been recorded, adding the new features to the map -
using previously identified landmarks as references. 
As previously stated, the Thymio II's extremely limited sensor capabilities
made the implementation of the map updating task practically impossible.

In order to correctly add new features to the internal map, a robotic agent
must first determine where the obstacle exists in a global context.
This is achieved by the robot utilising its sensors to locate the position
and distance of a previously observed map feature, and calculating the new
feature's distance from this known location.
The robot can then utilise the result of this calculation to add the estimated
location of the new feature to the approximated map.

Due to the Thymio II's sensors possessing a maximum effective range of around
6cm, the robot would not be able to accurately determine the location of an
obstacle that was placed more \~{}12cm away from any other environment feature.

An environment containing objects dispersed no more the 12cm away from one
other would be extremely impractical to testing a Thymio II SLAM solution:
\begin{itemize}
\item The close proximity of the obstacles would complex the process of
differentiating new features from previously observed features.
\item The Thymio II's dimensions measure 11cm by 11.2cm, making navigation of
\~{}12cm paths extremely difficult, to borderline impossible - due to the
unreliability of the Thymio's wheel motors.
\end{itemize}

As was the case of dead-reckoning, neither obstacle observation, or map
updating could be implemented on the physical robot due to constraints of the
native hardware.
It is feasible that a simulated implementation of the obstacle observation and
map updating could be produced, however alterations would first have
to be made to the simulator in order to increase the range of the artificial
Thymio II's sensors - which originally matched the documented range of the
real-world counterpart.

\subsection{Simulator}
\editnote{How the simulator was implemented/improved.}
