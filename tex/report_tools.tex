\section{Tools and Technologies}
%List and discuss the hard/software utilised in the project.
As is the case with many hardware centric studies, the choice of technologies
utilised by a project can drastically alter it outcome.
This project utilised a range of technologies and tools to implement the
proposed system, the details of which are provided here.

\subsection{Robot Platform}
A key concept of this study was the implementation of a SLAM solution
utilising inexpensive or primitive hardware components.
A Thymio II wireless robot was utilised for the implementation and testing of
the project's SLAM solution.
The Thymio II was chosen for this project due to its cost (typically available
for under £150) and its availability  - Edinburgh Napier University had
previously used the Thymio robots in a previous project.

The key features of the Tymio II wireless system are as follows:
\begin{itemize}
\item Nine infrared proximity sensors (five front mounted, two rear mounted,
and two ground facing), with an approximate range of 10cm.
\item 2.4GHz wireless module.
\item 3.7V, 1500mAh, Li-Po rechargeable battery.
\end{itemize}
Full specification of the Thymio robot is visible REF-HERE!

While the robot's cost point and accessibility were ideal for this
experimentation, there did exist a number of concerns in regards to Thymio
II's hardware capabilities.

A major concern was the robot's lack of proprioceptive sensors - sensors that
a robotic agent can utilise to detect how far, or at what angle it has been
moving.
