\section{Conclusion}
The goal of this project was to provide an answer the following research
\begin{quote}
``Is it feasible for a robot to map an environment, simulate an optimal route
through said environment, and then execute the simulated route - without
external hardware or equipment?'' 
\end{quote}
As was discussed in the Evaluation section of this report, it was not possible
to implement the SLAM solution required to fully explore and provide a
definitive answer the above question.
Had the robotic platform utilised in this study implemented some form of
odometry, or more accurate environmental sensors, it may have been possible to
produce a answer for the question as a whole, but as it stands, only
simulation portion of the question can be addressed.


As is made clear by the results goatherd from expire mention with robotic
simulators, it is possible for a simulation software to be made sufficiently
small or computationally inexpensive that is can be integrated directly with a
mobile robotic system.



As the use of non-deterministic behaviour is common, both with comer cal
and academic technologies, the benefits of systems being able to perform self
simulation and discovery based improvement are clear.

We additionally sought to implement simulation techniques that would be
applicable to the automated on-board testing of different strategies once the
environment had been mapped with SLAM. By considering three metrics applicable
to improving the ability to simulate many rounds, we were able to develop a new
simulator based on the code from the open source Aseba project's Playground
simulator. This new simulator was benchmarked to identify that it did in fact
consume fewer memory and CPU resources, allowing it to be hosted on smaller and
less powerful platforms such as those found on board or used alongside robots.
On more powerful platforms, the simulator can have many instances hosted in
parallel.

As well as this, the new simulator removed speed governing aspects ofthe
Playground, allowing it to complete fixed length/fixed goal simulations in a
shorter length of time, allowing more series executions in a given time frame
where software controlling simulations would seek to adapt its behaviour
depending on the outcome of previous simulations.
