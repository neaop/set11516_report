\section{Reflection}
This section of the report is intended to highlight the thoughts and opions of
the authors in regards to execution of the study, including; the management
and development techniques utilised during the project, as well as the personal
each of the authors personal thoughts of there own performance during the
course of the project.


\subsection{Project Scope}
In hindsight, the scope of this project was significantly larger than originally
anticipated.
This was in part due to the authors eagerness to participate in a thought
provoking and challenging investigation, and partly due to unknown number of
participants during the project's inception.
While it would have been wise to reduce the project's scope upon realisation
of there only being two participants, each member of the team had already made
headway with their respective investigation tasks, and it would have felt
wasteful or inefficient to simply stop progress on one half of the completed
work.
In reality, both team members had focused their effort on a single aspect of
the project - either the inexpensive SLAM solution, or the integrated robotic
simulator, it is likely that a suitably thorough investigation could have been
conducted - to the satisfaction of both team members.


\subsection{Project Management}
Due to the small size of the development team, an Agile management and
workflow system was utilised to aid in the direction of the project and its
deliverables.
The main agile technique employed was short, but regular work sprints - 
typically between one to two weeks, where each team member would work on
specific task, and aim to complete said task by the end of the sprint period.
Due to academic and personal commitments, it was not always possible for a task
to be completed within a sprint period, but as said academic commitments
occurred for both team members, there was a mutual understanding of these
occurrences.

An unseen side effect of the project's goals is that they split into two
distinct domains - the robotic SLAM implementation and the integrated
simulator, meaning each author could focus entirely on a single project
concept.
While splitting the investigation in this manner had its advantages, it was a
double edged sword, as each author could become entirely engrossed in a task
without needing, or requesting input from the other project participant.
This unintended task segregation could have been remedied by either reducing the
scope of the project to a single domain, or by increasing the number of
members on the team - thereby creating a scenario where multiple participants
would be working on a single project goal (and also reducing the overall work
load of each participant).

Even though the project's developmental tasks were completed individually,
regular team meetings were conducted as per the Agile management
recommendations.
Due to the shared academic commitments of both team members, and the shared
space in which they worked it was possible to conduct multiple meetings a
week, throughout the course of the project.
While these ad hoc meetings featured the benefit of regularity, the high rate
at which they were conducted reduced the overall importance of each individual
meeting - leading to lax meeting requirements or unfinished orders of business.
While there is no doubting the usefulness of regular meetings and discussions,
both for the planning and discussion of project aims and deliverables
progress - had a more formal meeting style been adopted each individual meeting
might have carried more weight - causing project advancement to have
progressed in a different style (for better or for worse).

In terms of tools utilised for the management of the team and the project,
Slack was used extensively for communication between the team members,
determining project goals, determination of sprint goals, and sharing of
literature or papers to aid in development.
Additionally, Git version control allowed each of the authors to keep track
each other progress and share code with one another.
While it was the case that most of the development task were segregated, there
were occasions where suggestions for improvements or bug-fixes were exchanged
utilising the source-code management tool.

Overall, the management of the project proved effective, while it might have
been the case that each author was overly confined to a single deliverable,
it did allow each developer to focus entirely on a single domain - effectively
letting each team member develop specialist knowledge in a sector of their
own interest.
Arguable the biggest short coming of the project's management was allowing
the scope of the investigation to remain as large as its inception - leading
to the effective production of two smaller projects, rather than a single
cohesive investigation.

\subsection{Personal Comments}
The following are the thoughts and comments of the project's authors,
in terms of their own personal performance during the project, and of the
project as a whole.

\subsubsection{Sam Dixon}
Being able to work on a project that covered such a wide range of topics and
technical domains was an exciting prospect, but in hindsight I realise Gareth
and myself certainly bit off more than we could chew.

I had previously worked briefly with integrated hardware and electronic as
part of university module, and I toughly enjoyed the experience.
For this project I was tasked with the implementation of the SLAM algorithm on
the physical Thymio robot.
Unfortunately, my previous work with electronics and sensing hardware did not
prepare me for the difficulties faced in this project - where every stage of
development was met with incompatible hardware problems.
In hindsight, I should have foreseen many of the problems I faced from
inspection of the robot's hardware specification - but my eagerness to resume
working with electronics, and the few papers I found indicating that some
semblance of SLAM was implementable on the Thymio II, blinded my reasoning -
leading to much frustration when I was unable to implement the simplest of
a SLAM solution's sub tasks.

In terms of project overall, I am pleased with the progress Gareth and myself
made - considering the multitude of tasks and domains we took on as a two
person team.
As has been stated in the project proper, of the amount of work we assigned
ourselves was not realistic - given the other coursework and personal
commitments we had in our lives.
Had we come to this realisation sooner, we might have been able to realign our
focus, and alter the overall outcome of the project.

Working with Gareth on this project was a fantastic experience - his technical
knowledge of multiple domains, and eagerness to answer questions and help
with problems I encounter greatly forwarded the progress on the project.
Had Gareth not shown interest in this project I probably would have considered
another topic for investigation - as his previous work and knowledge of
electronic proved invaluable to the progress of this study.

Overall, I though eely enjoyed working on this project, the range of topics to
be investigated and being pushed out of comfort area in terms of development
taught me a lot about the balancing of a project's goals, team expectations,
and divisor of labour.
If I were to reattempt this project with the knowledge I have now, I would
likely recommend one of the following actions; recruit more members to team,
limit the scope of the project to a single domain, or invest in a higher
quality robotic platform.

\subsubsection{Gareth Pulham}
% Talk about initial hopes
When the team was initially formed, we had high hopes about the amount of
ground we could cover, and our scope was initially very wide. Covering large
swathes of robotics and performance computing topics, we wanted to cover
genetic algorithms, robotic task planning, simultaneous localisation and
mapping, and GPU acceleration of the simulator. The total outcome would be
a complete system that could be strapped to the top of the Thymio II robots.
This system would provide SLAM capabilities to build up a picture of its
environment as well as a high performance GPU accelerated simulator that would
allow a GA system for evaluating different robotic control strategies. This
ambition was partly fed by our previous experience in areas related to what we
wanted to cover. In particular, we had experience in the analysis and
parallelisation of simulation tasks, as well as genetic approaches to finding
solutions in spaces such as planning.

% How they panned out
As we began our research into the theory required for this, as well as into the
Aseba codebase, it became clear to us that we would have to reduce the scope of
the project. Underlining this were the committments to other projects and tasks
academically and extracurricularly that reduced the amount of time we had
available to dedicate to the project. Fortunately, as our initial plan
comprised many smaller units of work used together, we could shed the higher
level components and still produce complete blocks of work that each had useful
outcomes. As mentioned, the codebase I personally worked on as well offered up
some surprises. Its structure and style of implementation was new to me and
it was difficult at first to construct a personal mental model of how the
Playground worked. Once this had been managed, benchmarking began. It was
frustrating following this process to find that the performance was not
ultimately bound by the physics calculations that we had hoped to target for
GPU acceleration, however, in completing investigation to the structure and
implementation, it was possible to develop new plans on how to improve the
performance.

% Concerns if any
Alongside the technical and logistical barriers that loomed as we moved from
the initial planning stages to implementation, there were also interpersonal
concerns. Although I've known Sam for some time, and appreciated the high
quality of work he had produced in the past, I had not worked directly with him
before in an academic environment. In practice this academic partnership worked
out well, and it was clear why Sam is regarded by his peers as an excellent
partner. His academic mindedness and research capability was a huge asset to
the project and I do not think that the outcomes acheived would be possible
with anyone else.

% Practical outcome
Those outcomes were ultimately something I think we can each be proud of. While
we didn't cover the same broad areas we started off hoping to, we were able to
realistically assess what we could produce and instead of spreading ourselves
too thin and producing a lower quality of work, we were able to focus on a
smaller number of topics in depth and at high quality.

% If I could do it again
If I were able to begin this project again from scratch, I would seek to reduce
the initial scope. The ambitions of the project in its earliest stages
with were very high, and while I think it was a useful exercise to be able to
research and replan as our understanding of the project landscape improved, I
hope that it's possible to use it to improve my estimation abilities and in
future be closer in my initial plan to the final outcome. I think that not just
is this a useful academic skill, allowing a researcher to more closely focus his
efforts and work in depth on what is a reasonable amount of work, I also think
it will turn out to be a useful skill in the workplace to be able to more
accurately guage the length of time assigned tasks might take.
