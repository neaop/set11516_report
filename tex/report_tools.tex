\section{Tools and Technologies}
%List and discuss the hard/software utilised in the project.
As is the case with many hardware centric studies, the choice of technologies
utilised by a project can drastically alter it outcome.
This project utilised a range of technologies and tools to implement the
proposed system, the details of which are provided here.

\subsection{Robot Platform}\label{tools_robot}
A key concept of this study was the implementation of a SLAM solution
utilising inexpensive or primitive hardware components.
A Thymio II wireless robot was utilised for the implementation and testing of
the project's SLAM solution.

The Thymio robots are a range of mobile robot products, designed to aid in the
education.
While aimed at younger audiences, the Thymio robots feature a range of
sensors and actuators that make them viable for a range of projects -
including research scenarios.

The Thymio II was chosen for this project due to its cost (typically available
for under £150) and its availability  - Edinburgh Napier University had
previously used the Thymio robots in a previous project.


The key features of the Thymio II wireless system are as follows:
\begin{itemize}
\item Nine infrared proximity sensors (five front mounted, two rear mounted,
and two ground facing), with an approximate range of 10cm.
\item 2.4GHz wireless module.
\item 3.7V, 1500mAh, Li-Po rechargeable battery.
\end{itemize}
Full specification of the Thymio II is visible at Tymio's website
\cite{thymio}.

While the robot's cost point and accessibility were ideal for this
experimentation, there did exist a number of concerns in regards to Thymio
II's hardware capabilities.

A major concern was the robot's lack of proprioceptive or odometric sensors -
sensors that a robotic agent can utilise to detect how far, or at what angle
it has moved.
A common odometry implementation is the use of rotary encoders on a robotic
agent's wheel.
The encoders can calculate the stance that the robot has travelled based on
the circumference of the wheels, and the number of rotations the wheels have
made.
While odometers is not required for the implementation of a SLAM solution, the
approximated localisation can significantly reduce the complexity of the
overall system.

The second major concern in regards to the Thymio II's specification was the
operational range of its infrared sensors.
Exteroceptive or external state sensors detect the environment around
them.
Their are a range of such sensors that can be utilised in SLAM solutions,
including but not limited to; laser scanners, ultrasonic sensors, infrared
detectors, and cameras.
A mobile exteroceptive sensor is the minimum requirement to implement a SLAM
solution, however the operational accuracy and range of said sensor can
greatly alter the challenge of implementation.
The Thymio II's detection of range of 10cm was concerning, but a number of
localisation and navigation studies had been conducted previously
\cite{Wang2016}, utilising the Thymio II, lessening the initial alarm in
regards to the sensors fidelity.

By default, the Thymio II's behaviour can be augmented by implementing
new systems in a bespoke programming language; Aseba.
The Aseba language's syntax bears many similarities to the Pascal and
Matlab languages - however it has been significantly simplified in order to
accommodate younger audiences.
Fortunately, the Thymio robots can support other programming languages,
assuming said language supports the D-Bus protocol - a low-level communication
system.
By choice of the project developers, it was opted that development of the
robot's behaviour software would be implemented in Python 2 - mainly due
familiarity with the language.

\subsection{Robot Simulator}
As one of the primary goals of this study was to evaluate the effectiveness of
robotic simulators, a suitable simulation software had to selected.
Fortunately, the Thymio robots are compatible with a existing simulation
service; Aseba Playground \cite{aseba-community}.

Aseba Playground provides a real-time, three dimensional rendering of
customisable environment, and allows a simulated Thymio robot to navigate,
sense, and interact with said environment - utilising the same sensor and
actuator configurations present on the real-world robot.

While not all features of the real-world robot are implemented within the
simulator, the ones required to implement a SLAM solution are provided, namely;

\begin{itemize}
\item The horizontal infrared detectors.
\item The wheel speed controls.
\end{itemize}

The Aseba simulator has been developed in C++, and is available under the GNU
Lesser General Public License, meaning the source-code can be viewed and
edited without financial commitment.

While the default configuration of Aseba Playground is quite computationally
expensive - mainly due to the visual rendering of environments and robots, the
availability of the source-code means it is feasible to reduce the
computational load of the application, and potentially allow integration with a
robotic agent.

The Aseba simulator is modular and the Playground comes as a collection of
several parts interconnected into a larger whole.

In particular, the following parts were of interest to this project:
\begin{itemize}
    \item The physics engine, which simulates the world as a whole, testing for
        collisions and tracking the locations and motion of agents within the
        simulation.
    \item The agent engine, which represents the robots both to the physics
        engine, and also their capabilities and sensing systems to systems that
        would seek to direct them.
    \item The GUI, which presents to the user an interface to control the
        simulation, as well as observe the results.
    \item Networking components, which present an interface for remote
        control of the simulation and the agents within it.
\end{itemize}
