\section{Background} \label{sec_background}
This section of the report provides a review and analysis of existing reports,
technical papers, and studies with the aim of providing context and evidence
of feasibility for this project.

The majority of the literature presented was aggregated prior to planning and
implementation of the simulated/real-world SLAM solution documented in this report,
and thus provided the inspiration and domain specific knowledge required to
carry out the project.


\subsection{Autonomous Robots and Vehicles}
SLAM, and autonomous navigation in general, has long been concept of interest
in a range of domains.
Academic, corporate, and government originations all have vested interest in
the advancement of self-navigating vehicles and robots, and a wide range of
such devices have already been developed and used to great affect.

One of the more common uses of autonomous navigation is in the exploration of
environments or areas that inhospitable to human life.

\subsubsection{Automotive}
Automotive automation is a key domain in which research has been focused in
recent years.
A wide range of organisation have all shown a great deal interest in the testing
and production of autonomous vehicles - well established car production companies
such as Mercedes Benz and BMW, technology companies like Google, and even
relatively new organisations like Tesla, Inc. and Uber Technologies Inc., have
all invested a significant amount of time and resources in the pursuit of
development, testing, and production of safe and reliable autonomous cars.

The concept of unmanned vehicles is not a new concept - the testing of radio
controlled cars dates back to mid 1920's, but recent research has had a much
stronger focus on the improvement of vehicles hazard detection and navigation
capabilities.

Due to the age of the concept of autonomous vehicles, a number of discrepancies
have arisen in regards to the correct terminology that should be utilised
within the field.
The name of the field has even come under security - arguments have been made
that the name `Automated Vehicles' would be more appropriate, as the term
autonomous implies the vehicle is entirely self governing, whereas in
reality most implementations of self-driving vehicles rely on some form of
external factors; magnetic strips, road markings, communications from other
vehicles, etc..
In order to aid with standardisation of terminology within the field, an
automotive standards body - SAE International defined a system in 2014
(J3016), allowing the classification of autonomous or automated vehicles
depending on the required user input or interaction with the vehicle.
The J3016 Autonomy Levels are visible in Appendix \ref{app-sae}. 



\subsubsection{Military}
One of the proposed domains where the use of autonomous agents or vehicles
could be advantageous is within military or combat scenarists - i.e. events
where rival or rouge agents may be attempting to delay, destroy, hijack, or
sabotage operations within an area.

While the use of autonomous, `intelligent' agents within combat situations
has number of ethical and morality concerns, the use of robotics by military
organisations is already underway.
Unmanned, but human controlled robotic platforms are heavily utilised to aid
with scouting or the disposal of hazardous objects - such as mines.
Further research has also been conducted into the feasibility of unmanned
vehicles aiding in the transport of of aid or materials through hazardous
fronts, and for use as evacuation or ambulance-like transport vehicles - to
help wounded personnel or aid in extraction scenarios.

There are a number of benefits to introducing autonomous agents within a military context.
One benefit is the reduced personnel, or personnel training, required to
operate specialist equipment.  
Lockheed Martin has been researching and developing a semi-autonomous
personnel carrier vehicle, utilised to safely transport human passengers to
and from dangerous areas.
The Autonomous Mobility Applique System (AMAS) can operate in a
`follow-the-leader` style fashion, where a convey of autonomous vehicles can
safely an accurately follow a designated, human operated, control vehicle.
Current test have shown that up to 7 AMAS vehicles can autonomous follow a
control vehicle, allowing the 14 navigation and driving personnel previously
required to perform other tasks during transportation.
Additionally, the AMASS can aid human driver by automatically detecting and
avoiding potential hazards that a human driver may miss - allowing personnel to operate
the vehicle at a reduced risk to themselves, passengers, and other vehicles.
Lokheed Martin also developed a smaller scale version of the AMAS, the Squad Mission
Support System (SMSS).

The SMSS utilises a similar `follow-the-leader' system to the larger AMAS, but
is designed to follow personnel on foot, carrying any tools or supplies that
might be required in the field.
The SMSS has already been deployed and used in multiple scenarios across the
globe, and has proven that autonomous mobile equipment can be a dependable
asset to combat personnel.

Lokheed Martin's motivation behind both the AMAS and SMSS designs was to
``improve the safety for the soldiers.''.
By producing equipment and vehicles that do not require human operation, the
number of personnel required to actively participate can be severely reduced,
removing the number of people required to enter hostile locations.
Additionally, the autonomous aid provided can servery reduce the physical and
mental load military personnel must endure in active duty.
It is crucial to note that the technologies outlined are not exclusive to
the military sector, in fact Lokheed has previously stated there goal of
reworking there autonomous platforms to be suitable for use in the consumer and
public domains, such as medical and search and rescue.
