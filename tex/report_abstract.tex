\begin{abstract}
%concise summary of the study
The paper documents the processes of investigation, implementation, and
evaluation of Simultaneous Localisation and Mapping solution utilising an
inexpensive robot and a simulation system.
The goal of the project was to implement a mapping algorithm on the physical
robot, which would explore and document environment.
Once a map of an area had been generated the robot would be able to utilise
the simulation software to calculate optimal routes through the environment,
before executing said movements in the real world.
A Thymio II wireless robot and the Aseba simulator were utilised as
development platforms due to low cost and availability.
Due to the Thymio II's lack of odometry and limited sensors, it was not
possible to implement localisation and mapping algorithms of sufficient
accuracy to enable the robot navigate and document an area.
The simulator was modified to reduce the required computational cost of
operation and allow multiple simulators to execute in parallel - enabling
systems of limited computational power to perform simulated robotic
navigation.
The study concludes that while it would be possible to implement a
localisation and mapping solution on an inexpensive robot, said robot would
require either; some form odometry or accurate dead-reckoning, or
distance sensors of sufficient accuracy and with an effective range
significantly greater than the dimensions of the robot carrying them -
features not present in the Thymio II system.
The modified simulator's runtime requirements were made sufficiently small that
it could be executed on device that would impede the Thymio II's mobility,
potentially allowing the robot to simulate navigation of the mapped area.
While the study did manage to achieve the all the initially planned
deliverables, details of limitations and potential solutions are provided.
\end{abstract}
