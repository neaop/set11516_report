\pagestyle{fancy}
\section{Introduction} \label{sec_introduction}
This report serves to document and present the processes, methods, and
technologies utilised in the development of a combined real-world and
simulated implementation of a robotic Simultaneous Location and Mapping
system - allowing a physical robot to map environment, simulate an optimal
route through said environment, and then execute the simulated movement in the
real world.

The autonomous mapping of, and subsequent movement around unknown environments
via robotic agents has long been area of interest, and numerous
implementations of the concept are readily available in commercial products
such as the Roomba autonomous vacuum cleaner, or Honda's  robotic lawnmower,
Miimo.

Most current implementations of self mapping robots require some form of
specialist or dedicated sensors to accurately navigate and record a new
environment.
While use of this specialist equipment has proven effective when attempting
to implement  mapping and navigation solutions, the financial costs of
developing such systems can often make them unfeasible.
One of the goals for this project was to investigate the possibility of
implementing a localisation and mapping solution using a robot lacking
the specialist hardware often present in commercial and research
implementations.

The use of simulation software to test, evolute, and develop robotic behaviour
offers a number of benefits, including reduced development costs and time.
One of the benefits of simulation is that it can be conducted remotely, without
needing access to a robots physical hardware.
One of the aims of this study was to investigate the advantages of integrating
simulation software directly with the robotic hardware - effective allowing a
robotic agent to conduct its own simulation routines.
The advantages of a combined robotic agent/simulation system is the potential
for a robot to calculate solutions to encountered problems without the need to
contact an external system to conduct said simulations.
Use cases for such a system would be robotic agents in environments that are
able to regularly `phone home' or establish a connection to  external
simulation systems - planetary rovers and deep sea submersibles  are both
examples of systems that may become uncontainable for extended periods of time.


\subsection{Main Topics}
This project explored and utilised a wide array of technologies and techniques
to research, implement and test the proposed system.
The key concepts of the project are briefly described in order to give some
insight to some of the approaches utilised .

\subsubsection{SLAM}
Simultaneous Location and Mapping (SLAM) is computational problem relating
to the tracking of an agent's location in an environment while concurrently
constructing a map of the environment that contains the agent.

The primary difficulty to producing a solution to the SLAM problem is
that SLAM is a circular problem: in order to accurately determine a location
within an area, a map of said area is required - and in order to create a map
of environment, an agent is required to already be aware of it's location in the
environment.

A number of algorithms and approaches have been theorised and developed to
solve the problem of SLAM.
The choice of which SLAM solution implementation to utilise depends heavily
on the environment being mapped, the equipment available to the agent, and the
manner in which the map is to be produced.
Further discussion of SLAM and SLAM solutions is provided in Section REF HERE!


\subsubsection{Routing}
Routing problems consist of a simple premise: ``What is the optimal path to
take to travel from point A to point B''.
The main difficulties present in the routing problem domain typically consist of
defining what 'optimal' means, and selecting the single best solution from the
huge array of possible answers.

In terms of optimisation, the `best' solution to a routing problem may be the path that takes the
least time, covers the smallest amount of distance, requires the least
resources, or any combination of the previous and other metrics.

Much like SLAM, a wide range of algorithms and solutions have been researched
and evaluated to solve the routing problem - a selection of which are
discussed in Section REF HERE!


\subsubsection{Robotic Simulation}
Robotic simulators allow a robot's hardware or behaviour to be tested and
evaluated without relying upon a physical robot.
Simulators provide a number of benefits when researching, developing, or
evaluating new solutions to a problem domain, or when time or resource
expensive tests or scenarios need to be replicated or repeated to prove a
concept or design.

Simulators come in various styles and appearance; some rendering a virtual
three-dimensional  environment and robots, others offer no such visualisation,
requiring a much smaller runtime cost - allowing multiple robots or
environments to be tested simultaneously.

Regardless of the type of simulator utilised, the technology allows for
consistent, repeatable tests and scenarios to be conducted, and testing
parameters to be altered at a reduced cost - both in terms of time and
hardware requirements.
Descriptions of the simulation tools and techniques utilised in this project
are discussed further in Section REF HERE!.


\subsubsection{Research questions} \label{sec_res_q}
In order to keep the scope of this project somewhat reasonable, and as an
attempt to prevent `feature-creep' distracting from the primary project aims,
the following research question was proposed:
\begin{quote}
    ``Is it feasible for a robot to map an environment, simulate an optimal
    route through said environment, and then execute the simulated route -
    without external hardware or equipment?''
\end{quote}

The previous research question can be further broken down into the following
sub-queries:

\begin{quote}
    ``Is it possible to implement a SLAM solution using a robot with limited or
    primitive hardware?''
\end{quote}

\begin{quote}
    ``Is it possible for a robot to execute its own lightweight simulator using
    integrated hardware?''
\end{quote}

Each of the proposed sub-queries are key areas of research, both in academia
and industry.
Being able to implement solutions to complex problems using simple hardware
can greatly improve the cost efficiency of an organisation - leading to
finances being available for more costly projects or research. 
Additionally, implementing lightweight or computationally inexpensive software
solutions provides multiple benefits - such as embedding software onto small,
dedicated systems, or simply reducing the amount memory a system requires to
implement and execute a software solution. 

The research questions presented previously were utilised throughout all
stages of this project, in order to ensure effort and time was utilised
efferently upon key aspects of development and research - and ultimately
forwarding the progress of the project.

\subsection{Report Structure}
This report is divided appropriately;

\begin{itemize}

\item A background review and analysis of existing literature, white papers, and
documentation; in order to both prove the feasibility of and justify this
project and the techniques and technologies utilised in its execution.

\item Description and justification for the choice of tools, hardware, and
technologies utilised throughout the project.

\item A description of the methodologies utilised during the
development and testing of the project and it's systems.

\item Details of the metrics used to evaluate the project.

\item Presentation of findings made during the implantation or testing.

\item An evaluation of the techniques used, and the results of the project.

\item A summery concluding the key findings of the project and potential future
work.

\item The personal reflections of the authors, their opinions of project, and any
improvements or changes that they have observed in hindsight.

\end{itemize}
