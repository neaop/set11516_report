\section{Evaluation}
This section of the report aims to  discuss and describe the information that
can be gained from the results goatherd in the previous section.
Evaluation of results is a critical step in determination of susses of a
study, and ultimately answering the research question originally posed in
Section \ref{sec_res_q}.

\subsection{Robot}
Due to short comings of the Thymio II platform described in Section
\ref{meth_slam}, it was not possible to implement the required subtasks or
models needed to produce a valid SLAM solution.
As a result of this lack of implementation, it is not possible to provide a
formal evaluation of the system.

While the evaluation  of the SLAM implementation is not possible, it is still
possible to discuss the metrics that may have been utilised to quantify the
SLAM implementation.

As per previous studies conducted in the field of autonomous navigation, one of
the most common metrics of evaluation is a simple comparison of the robot's
estimated location of map features and itself, and the physical location of
said features.
It is crucial to remember that SLAM solutions are approximation techniques,
and as such a certain degree of error is expected within an robot's
estimations.

A second common metric or evaluation SLAM solutions is to measure the time
take for a robotic agent to map a certain percentage of an environment, or to
find a number of specific landmarks within said environment.
Timed metrics are often an unreliable evaluation technique, as robotic agents
can produce artificially high results with low accuracy SLAM solutions, by
simply moving at a faster speed through an environment.

Much like a SLAM solution itself, the techniques utilised in the evaluation of
an implementation must be tailored to system, the sensors utilised, the manner
of estimation, the type of map produced, and the specification of the robot.

One of the two sub-queries posed as a research question for this study was:
\begin{quote}
``Is it possible to implement a SLAM solution using a robot with limited or
primitive hardware?''
\end{quote}
While this study's inability is indicative of a negative answer to this
question, much of the literature reviewed in Section \ref{sec_slam} reveals
that implementation of SLAM solutions on primitive robotic platforms is in
fact possible.

In hindsight the original sub-query can be consider vague or leading, a more
suitable research question would have been:

\begin{quote}
``What are the hardware limitations to implementing a SLAM solution on a
mobile robotic platform?''
\end{quote}

While this revised research query cannot be answered by the results gained by
this study, it is certainly a more appropriate question, and a suitable topic
for a future investigation.


\subsection{Simulator}
\subsubsection{Memory Consumption}
\editnote{Graph Playground vs Microenki RAM}
\editnote{Compare and contrast memory consumption}

\subsubsection{CPU Consumption}
\editnote{Graph Playground vs Microenki RAM}
\editnote{Compare and contrast memory consumption}

\subsubsection{Simulation Ratio}
\editnote{Graph Playground vs Microenki RAM}
\editnote{Compare and contrast memory consumption}
